% Lectures/lec_22.tex
\lecture{20}{}{Rocket Soft Landing and Convexification}

\begin{prev}
	In previous lectures, we covered Stochastic Optimal Control (LQG) and State Estimation (Kalman Filter).
	Today, we shift gears to a specific, high-impact application of optimal control: \textbf{Rocket Soft Landing}.
	This problem motivated a major breakthrough in trajectory optimization known as "Lossless Convexification," which allows us to solve non-convex powered descent problems using convex solvers (SOCP) with global optimality guarantees.
\end{prev}

\subsection{The Rocket Soft-Landing Problem}

The goal is to navigate a rocket from an initial state to a target landing site (usually at \(z=0\)) with zero terminal velocity, while minimizing fuel consumption and respecting physical constraints.

\begin{itemize}
	\item \textbf{Objective:} Minimize fuel consumption (maximize remaining mass) or landing error.
	\item \textbf{Constraints:} Thrust limits (min/max), glide slope (pointing angle), safety zones.
	\item \textbf{Real-world Examples:}
	      \begin{itemize}
		      \item NASA Curiosity "Sky Crane" (Mars, 2012).
		      \item SpaceX Falcon 9 Boostback and Landing.
		      \item NASA Perseverance with Terrain Relative Navigation (TRN) (Mars, 2021).
		      \item SpaceX Starship.
	      \end{itemize}
\end{itemize}

\subsection{The "Full Stack" Control Architecture}

Landing a rocket requires a hierarchical control approach to handle different timescales and physics.


\begin{enumerate}
	\item \textbf{State Estimation:}
	      \begin{itemize}
		      \item \textbf{Earth (SpaceX):} GPS + IMU + Altimeter. Provides accurate position (\(\sim\) meters), velocity, and attitude.
		      \item \textbf{Mars (NASA):} No GPS. Relies on IMU integration, Radar Altimeter, and Vision (Terrain Relative Navigation) to avoid boulders and achieve \(\sim 30\)m accuracy.
	      \end{itemize}
	\item \textbf{High-Level Position Controller (Guidance):}
	      \begin{itemize}
		      \item Solves the trajectory optimization problem.
		      \item Reasons about safety, fuel, and thrust limits.
		      \item Treats the rocket as a \textbf{point mass}.
		      \item Runs at low frequency (\(\sim 1\)-\(10\) Hz).
		      \item Outputs: Desired acceleration vector (which implies thrust magnitude and pointing direction).
	      \end{itemize}
	\item \textbf{Low-Level Attitude Controller:}
	      \begin{itemize}
		      \item Tracks the desired pointing direction commanded by the guidance layer.
		      \item Deals with rigid body dynamics, aerodynamics, and disturbances.
		      \item Handles \textbf{fluid slosh} (often modeled as a pendulum) and \textbf{flexible modes} (bending of the rocket body, handled via notch filters).
		      \item Runs at high frequency (\(\sim 50\)-\(100\) Hz).
	      \end{itemize}
\end{enumerate}

\subsection{Rocket Dynamics}

For the guidance problem, we typically use a 3-DOF point mass model with variable mass.

\begin{align}
	\dot{\vec{r}}(t) &= \vec{v}(t) \\
	\dot{\vec{v}}(t) &= \vec{g} + \frac{\vec{T}(t)}{m(t)} \\
	\dot{m}(t) &= -\alpha \|\vec{T}(t)\|
\end{align}
where:
\begin{itemize}
	\item \(\vec{r}, \vec{v} \in \mathbb{R}^3\) are position and velocity.
	\item \(\vec{g} = [0, 0, -g]^\top\) is gravity.
	\item \(\vec{T}\) is the thrust vector in the world frame.
	\item \(\alpha\) is the mass depletion rate coefficient (related to specific impulse \(I_{sp}\)).
\end{itemize}

\begin{note}
	Fuel mass is significant (can be \(80\%\) of initial mass), so \(\dot{m}\) cannot be ignored.
	Aerodynamic forces are often ignored in the final landing burn phase (Mars atmosphere is thin, or speeds are low).
\end{note}

\subsection{Convex Relaxation}

The dynamics above are non-linear due to the \(\frac{\vec{T}}{m}\) term, and the thrust constraints can be non-convex.
To solve this efficiently (online), we need to cast it as a convex problem.

\begin{intuition}[Convex Relaxation]
	Consider an optimization problem constrained to the boundary of a set, \(S_1 = \{ \vec{x} \mid \|\vec{x}\| = 1 \}\). This is non-convex (a spherical shell).
	We can \textbf{relax} the constraint to the convex set \(S_2 = \{ \vec{x} \mid \|\vec{x}\| \le 1 \}\) (the solid ball).
	\[
		\min_{\vec{x}} \vec{c}^\top \vec{x} \quad \text{s.t.} \quad \|\vec{x}\| = 1 \implies \min_{\vec{x}} \vec{c}^\top \vec{x} \quad \text{s.t.} \quad \|\vec{x}\| \le 1
	\]
	If the cost function "pushes" the solution to the boundary (e.g., minimizing \(\vec{c}^\top \vec{x}\) pushes \(\vec{x}\) as far as possible in the direction \(-\vec{c}\)), the optimal solution to the relaxed problem will satisfy \(\|\vec{x}^*\| = 1\).
	When this happens, we call the relaxation \textbf{tight} or \textbf{lossless}. We solved the easier convex problem but got the solution to the hard non-convex one.
\end{intuition}

\subsection{Convexification of Thrust Constraints}

We need to constrain the thrust vector \(\vec{T}\).

\subsubsection{1. Maximum Thrust (Convex)}
We have an upper limit on engine power:
\[ \|\vec{T}\| \le T_{\max} \]
This describes a solid sphere (or ball), which is a \textbf{convex} set (specifically, a Second-Order Cone constraint).

\subsubsection{2. Glide Slope / Pointing Angle (Convex)}
The rocket engine cannot point too far away from the vertical (to avoid tipping or sensor occlusion).
Let \(\hat{\vec{n}}\) be the vertical axis. We require the angle \(\theta\) between \(\vec{T}\) and \(\hat{\vec{n}}\) to be small:
\[ \theta \le \theta_{\max} \implies \cos \theta \ge \cos \theta_{\max} \]
Using the dot product \(\vec{T} \cdot \hat{\vec{n}} = \|\vec{T}\| \cos \theta\):
\[ \hat{\vec{n}}^\top \vec{T} \ge \|\vec{T}\| \cos \theta_{\max} \]
This is a \textbf{Second-Order Cone (SOC)} constraint, which is convex.

\subsubsection{3. Minimum Thrust (Non-Convex)}
Rocket engines cannot throttle down to zero continuously; they have a minimum stable thrust level (or they flame out).
\[ \|\vec{T}\| \ge T_{\min} \]
This removes a small ball from the center of the feasible thrust set.
The resulting set (a hollow shell between \(T_{\min}\) and \(T_{\max}\)) is \textbf{non-convex}.
\begin{center}
	% Figures/inequality_constraint.tex
\begin{tikzpicture}[
    font=\sansmath\sffamily,
    scale=1.5
]
    % Axes
    \draw[-Latex, thick] (-2.5,0) -- (2.5,0) node[below left] {$x_1$};
    \draw[-Latex, thick] (0,-2) -- (0,2.5) node[below left] {$x_2$};

    % f(x) contours
    \draw[NavyBlue] (0,0) circle (0.5);
    \draw[NavyBlue] (0,0) circle (1);
    \draw[NavyBlue] (0,0) circle (1.5);

    % Inequality constraint c(x) >= 0
    \draw[thick, Red] plot[domain=-2.5:1, samples=100] (\x, {(\x+1.5)^2 - 1});
    \node[Red] at (-2, 2) {$c(\vec{x}) = 0$};

    % Feasible region shading
    \begin{scope}
        \clip plot[domain=-2.5:1, samples=100] (\x, {(\x+1.5)^2 - 1}) |- (2.5,-2) -- (-2.5,-2) -- cycle;
        \fill[Red, opacity=0.1] (-2.5,-2) rectangle (2.5,2.5);
    \end{scope}
    \node[Red, font=\small] at (1,1.5) {Feasible Region $c(\vec{x}) \geq 0$};

    % Optimal point
    \coordinate (opt) at (-0.29, 0.46);
    \fill[black] (opt) circle (1.5pt) node[above right, font=\small]{Active Constraint};

    % Gradients at active constraint
    \draw[-Latex, thick, ForestGreen] (opt) --++ (-0.2, 0.3) node[above] {$\nabla f$};
    \draw[-Latex, thick, Orange] (opt) --++ (-0.2, 0.3) node[below right, xshift=-2pt] {$\nabla c$};

    % Inactive constraint example
    \coordinate (inactive) at (1.5,0);
    \fill[gray] (inactive) circle (1.5pt) node[below, font=\small]{Inactive Constraint};
    \draw[-Latex, thick, ForestGreen] (inactive) -- (inactive) node[above=2pt] {$\nabla f=0$};

\end{tikzpicture}
 % Reusing a generic constraint figure placeholder or draw a donut
\end{center}
This non-convexity prevents the direct use of convex solvers like SOCP.

\subsection{Lossless Convexification}

Acikmese et al. (2007) introduced a "slack variable" trick to convexify the minimum thrust constraint.

\paragraph{Step 1: Change of Variables.}
Introduce a new scalar decision variable \(\Gamma(t)\) representing the thrust magnitude.
We impose constraints on \(\vec{T}\) and \(\Gamma\):
\begin{enumerate}
	\item \(\|\vec{T}\| = \Gamma\) (Non-convex equality constraint: \(\vec{T}\) must lie \emph{on} the cone defined by \(\Gamma\)).
	\item \(T_{\min} \le \Gamma \le T_{\max}\) (Linear/Convex box constraint).
	\item \(\hat{\vec{n}}^\top \vec{T} \ge \Gamma \cos(\theta_{\max})\) (Linear/Convex cone constraint in \(\vec{T}, \Gamma\)).
\end{enumerate}

\paragraph{Step 2: Relaxation.}
We relax the non-convex equality (1) to an inequality:
\[
	\|\vec{T}\| \le \Gamma
\]
This is a convex Second-Order Cone constraint.

\paragraph{Step 3: The Result.}
The final optimization problem is:
\begin{align*}
	\min & \quad \int_{0}^{t_f} \Gamma(t) \, dt \quad (\text{Minimize fuel/thrust magnitude}) \\
	\text{s.t.} & \quad \text{Dynamics (linearized or discretized)} \\
	& \quad \|\vec{T}(t)\| \le \Gamma(t) \\
	& \quad T_{\min} \le \Gamma(t) \le T_{\max} \\
	& \quad \hat{\vec{n}}^\top \vec{T}(t) \ge \Gamma(t) \cos(\theta_{\max})
\end{align*}
\textbf{Theorem (Lossless Convexification):} For the minimum fuel problem, the optimal solution to the relaxed problem \textbf{always} satisfies \(\|\vec{T}^*(t)\| = \Gamma^*(t)\).
Therefore, the relaxation is tight, and we can solve the non-convex rocket landing problem using efficient convex solvers (like ECOS or OSQP) in real-time.

\begin{note}
	This works because minimizing \(\int \Gamma \, dt\) encourages \(\Gamma\) to be as small as possible. Since \(\Gamma\) is bounded from below by \(\|\vec{T}\|\), the optimizer squeezes \(\Gamma\) down until it hits the boundary \(\|\vec{T}\|\).
\end{note}
\newpage
