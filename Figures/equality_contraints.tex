% Figures/equality_constraint.tex
\begin{tikzpicture}[
    font=\sansmath\sffamily,
    scale=1.5
]
    % Axes
    \draw[-Latex, thick] (-2.5,0) -- (2.5,0) node[below left] {$x_1$};
    \draw[-Latex, thick] (0,-2) -- (0,2.5) node[below left] {$x_2$};

    % f(x) contours
    \draw[NavyBlue] (0,0) circle (0.5);
    \draw[NavyBlue] (0,0) circle (1);
    \draw[NavyBlue] (0,0) circle (1.5);
    \node[NavyBlue, rotate=-45] at (1.2, 1.2) {Level curves of $f(\vec{x})$};


    % Equality constraint c(x) = 0
    \draw[thick, Red] plot[domain=-2.5:1, samples=100] (\x, {(\x+1.5)^2 - 1});
    \node[Red] at (-2, 2) {$c(\vec{x}) = 0$};

    % Optimal point
    \coordinate (opt) at (-0.29, 0.46);
    \fill[black] (opt) circle (1.5pt);

    % Gradient vectors
    % ∇f
    \draw[-Latex, thick, ForestGreen] (opt) --++ (-0.2, 0.3) node[above] {$\nabla f$};
    % ∇c
    \draw[-Latex, thick, Orange] (opt) --++ (0.2, -0.3) node[below] {$\nabla c$};

    \node[font=\small] at (1,-1.5) {At optimum, $\nabla f \propto -\nabla c$};
\end{tikzpicture}
